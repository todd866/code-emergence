\documentclass[12pt]{article}
\usepackage{amsmath}
\usepackage{amssymb}
\usepackage{geometry}
\usepackage{graphicx}
\graphicspath{{supporting_files/}}
\usepackage{booktabs}
\usepackage[round]{natbib}  % Author-date citations (Harvard style)
\bibliographystyle{abbrvnat}  % Abbreviated author-date format
\usepackage{hyperref}
\geometry{margin=1in}
\usepackage{setspace}
\doublespacing

% Macros
\newcommand{\Neff}{N_{\text{eff}}}
\newcommand{\Deff}{D_{\text{eff}}}

\title{Low-Dimensional Codes Constrain High-Dimensional Biological Dynamics}

\author{Ian Todd\\
Sydney Medical School\\
University of Sydney\\
Sydney, NSW, Australia\\
\texttt{itod2305@uni.sydney.edu.au}}

\date{\today}

\begin{document}

\maketitle

\begin{abstract}
Biological systems exhibit persistent organization despite operating in high-dimensional, noisy, and finite-time dynamical regimes. While order parameters and attractor dynamics explain pattern formation within single systems, they do not account for how biological organization is stabilized through low-dimensional interfaces such as genetic, regulatory, or homeostatic codes. Here we propose a minimal dynamical mechanism by which low-dimensional codes emerge as stabilizing constraints between coupled high-dimensional systems. We model two locally coupled phase fields, representing an environment and a responding system, where interaction is restricted to a low-dimensional projection of the environmental state. Using a Fourier bottleneck to control code bandwidth, we show that reducing coupling dimensionality induces a systematic collapse in the responding system's behavioral complexity while preserving bounded tracking of the driving system. Crucially, this collapse requires structured projections that capture coherent macroscopic degrees of freedom; random $k$-mode projections of the same dimensionality fail to constrain complexity. The code constrains dynamics without enabling prediction or reconstruction of the full environmental microstate. Parameter sweeps suggest a trade-off between stability and behavioral richness: intermediate code dimensionality maintains bounded tracking under noise, while excessive compression produces rigid, low-complexity behavior. These results provide a dynamical explanation for the ubiquity of low-dimensional coding structures in biology as constraint-forming devices that preserve viability in high-dimensional systems, rather than as predictive representations.
\end{abstract}

\noindent\textbf{Keywords:} Code formation, dimensional constraint, non-ergodicity, Kuramoto oscillators, stabilizing constraints, biological organization

\section{Introduction}

Biological systems operate in high-dimensional state spaces while maintaining persistent, viable organization over time. This persistence is remarkable: high-dimensional systems cannot exhaustively explore their state space in finite time, yet organisms remain within viable regions across generations. How is this stability achieved?

Existing theoretical frameworks address related questions but leave a gap. Synergetics and order parameter theory explain how fast variables become ``slaved'' to slow modes within a single system \citep{haken1983,haken2004}, producing pattern formation and dimensional reduction. Developmental canalization describes robustness of phenotypes to genetic and environmental perturbation \citep{waddington1942,waddington1957,kerszberg1989,siegal2002,felix2015}. Information-theoretic approaches model compression that preserves predictive relevance \citep{tishby1999,still2012}. Markov blanket formalisms describe boundaries that separate internal from external states \citep{palacios2020,friston2013,kirchhoff2018}. Computational mechanics characterizes intrinsic computation in dynamical systems \citep{crutchfield1989,shalizi2001b}. Constraint-based approaches emphasize organizational closure \citep{mossio2009,montevil2015,moreno2015}.

However, none of these directly address the mechanism by which \textit{low-dimensional codes emerge as the sole coupling channel between two high-dimensional dynamical systems}, nor how bandwidth limitations in such codes systematically constrain the behavioral complexity of the responding system.

\subsection{Contribution}

We propose that low-dimensional codes function as \textit{stabilizing constraints} rather than predictive representations. Formally, a code is an \textit{interface variable}: a projection $C_k = \mathcal{P}_k(X)$ that is causally efficacious for the responding system while being information-lossy about the driving system's microstate. High causal influence does not imply high mutual information with the microstate; the code shapes dynamics without enabling reconstruction.

Specifically:

\begin{enumerate}
\item A code is a low-dimensional projection $C_k = \mathcal{P}_k(X)$ of a high-dimensional system state $X(t)$, where $\dim(C_k) \ll \dim(X)$.
\item When system $B$ is coupled to system $A$ only through $C_k(A)$, the code acts as a constraint that shapes $B$'s accessible trajectories.
\item Reducing code bandwidth $k$ induces systematic collapse in $B$'s behavioral complexity while $A$'s complexity remains unchanged.
\item This provides a dynamical explanation for the ubiquity of low-dimensional coding structures in biology.
\end{enumerate}

We demonstrate this mechanism using a minimal model: two locally coupled phase fields (Kuramoto lattices \citep{kuramoto1984,acebron2005,strogatz2000}) where the responding system receives only a truncated Fourier representation of the driving system's state. While Fourier modes are convenient analytically, control experiments reveal that the constraining effect is not about projection dimensionality per se: random $k$-mode projections of the same dimension fail to induce complexity collapse. The constraint requires \textit{structured} projections that capture spatially coherent patterns while discarding incoherent microscale fluctuations. Our claim is therefore not ``any $k$-dimensional bottleneck stabilizes,'' but rather ``stabilizing codes must transmit coherent macroscopic degrees of freedom; bandwidth limitation is necessary but not sufficient.'' This suggests that biological codes must be similarly structured---capturing dynamically relevant collective variables rather than arbitrary dimensional reductions.

\section{Methods}

\subsection{Model Overview}

We consider two coupled high-dimensional dynamical systems: a driving system $A$ and a responding system $B$ (Figure~\ref{fig:schematic}). Each system consists of a one-dimensional lattice of $N$ locally coupled phase oscillators. System $A$ evolves autonomously, while system $B$ attempts to track $A$ through a low-dimensional projection of $A$'s state. Direct access to the full state of $A$ is not permitted.

\begin{figure}[ht]
\centering
\includegraphics[width=0.85\textwidth]{fig_schematic.pdf}
\caption{Schematic of the coupled system architecture. The driving system $A$ projects its state onto a low-dimensional code $C_k$ containing only $k$ Fourier modes. The responding system $B$ couples to $A$ only through this bandwidth-limited code, inducing complexity collapse in $B$ that scales with the constraint bandwidth.}
\label{fig:schematic}
\end{figure}

\subsection{Driving System Dynamics}

The driving system is defined by phases $\theta^A_i(t) \in (-\pi, \pi]$, $i = 1, \ldots, N$, evolving according to a locally coupled Kuramoto model \citep{kuramoto1984,pikovsky2001}:
\begin{equation}
\dot{\theta}^A_i = \omega_i + K \sum_{j \in \mathcal{N}(i)} \sin(\theta^A_j - \theta^A_i) + \eta^A_i(t),
\label{eq:systemA}
\end{equation}
where $\omega_i$ are intrinsic frequencies drawn from a narrow distribution, $K$ is the coupling strength, $\mathcal{N}(i)$ denotes nearest neighbors on a ring lattice, and $\eta^A_i(t)$ is Gaussian white noise with variance $\sigma^2$.

\subsection{Low-Dimensional Constraint (Code)}

At each time step, the state of system $A$ is projected onto a low-dimensional constraint via a truncated Fourier representation. Defining the complex phase field
\begin{equation}
z^A_i = e^{i\theta^A_i},
\end{equation}
the constraint of dimension $k$ is given by the first $k$ Fourier coefficients:
\begin{equation}
C_m = \frac{1}{N} \sum_{i=1}^{N} z^A_i e^{-i 2\pi m i / N}, \quad m = 0, \ldots, k.
\label{eq:fourier}
\end{equation}
A reconstructed low-pass approximation $\hat{\theta}^A_i$ is obtained by inverse transforming the truncated spectrum. This reconstruction preserves coarse spatial structure while discarding higher-frequency detail.

\subsection{Responding System Dynamics}

The responding system $B$ evolves according to:
\begin{equation}
\dot{\theta}^B_i = \omega_i + K \sum_{j \in \mathcal{N}(i)} \sin(\theta^B_j - \theta^B_i) + \lambda \sin(\hat{\theta}^A_i - \theta^B_i) + \eta^B_i(t),
\label{eq:systemB}
\end{equation}
where $\lambda$ controls the strength of coupling to the low-dimensional constraint. System $B$ has no access to the full state of system $A$, only to the reconstructed field $\hat{\theta}^A$.

\subsection{Quantifying Behavioral Complexity}

Behavioral complexity is quantified using the spectral entropy of each system's phase field. The power spectrum of the Fourier coefficients of $z_i$ is normalized to obtain a probability distribution $p_m$, from which the spectral entropy is computed:
\begin{equation}
H = -\sum_m p_m \log p_m.
\end{equation}
The effective dimensionality is defined as $\Neff = e^H$, representing the number of active modes contributing to the dynamics \citep{inouye1991,grassberger1983}.

\subsection{Mismatch and Stability Measures}

Mismatch between systems is quantified as the mean circular distance:
\begin{equation}
\Delta = \frac{1}{N} \sum_i \left| \sin\left(\frac{\theta^A_i - \theta^B_i}{2}\right) \right|.
\end{equation}
Stability is assessed via long-time variance of $\theta^B$ and residence within bounded regions of phase space under noise.

\subsection{Parameter Sweeps}

Simulations were conducted across a range of constraint dimensions $k$ and coupling strengths $\lambda$, holding other parameters fixed ($N = 64$, $K = 0.5$, $\sigma = 0.3$, $dt = 0.1$). For each parameter combination, metrics were averaged over 15 independent realizations after transient dynamics had decayed (500 burn-in steps, 500 measurement steps).

\section{Results}

\subsection{Complexity Collapse Under Bandwidth Reduction}

Figure~\ref{fig:complexity} shows the central result: as code bandwidth $k$ decreases, the effective dimensionality of system $B$ decreases systematically, while system $A$'s complexity remains approximately constant.

At $k = 1$ (single Fourier mode), $\Neff(B) = 9.4 \pm 0.7$, compared to $\Neff(B) = 14.8 \pm 0.7$ at $k = 32$. Meanwhile, $\Neff(A) = 13.2 \pm 0.4$ across all values of $k$. The relationship between $k$ and $\Neff(B)$ is approximately monotonic, with some fluctuation at intermediate $k$ (visible in Figure~\ref{fig:complexity}) that falls within error bars and likely reflects finite-sample variance and mode-locking effects at specific bandwidths. The central finding is robust: the bottleneck constrains only the responding system; the driving system's dynamics are unaffected by how it is observed.

\begin{figure}[ht]
\centering
\includegraphics[width=0.8\textwidth]{fig_complexity.pdf}
\caption{Effective dimensionality as a function of code bandwidth. The driving system $A$ (gray) maintains constant complexity regardless of $k$, while the responding system $B$ (blue) exhibits systematic complexity collapse as bandwidth decreases. Error bars show standard error over 15 independent trials.}
\label{fig:complexity}
\end{figure}

\subsection{Mismatch Remains Bounded Despite Complexity Collapse}

Mismatch between systems $A$ and $B$ increases modestly as code bandwidth decreases (Figure~\ref{fig:mismatch}): $\Delta = 0.38$ at $k = 32$ versus $\Delta = 0.46$ at $k = 1$. This modest variation ($\sim$20\%) contrasts with the dramatic complexity collapse ($\sim$57\% reduction in $\Neff(B)$).

The interpretation is subtle but important. Mismatch measures phase distance, not information content. At low $k$, system $B$ tracks the \textit{code} faithfully---it follows $A$'s coarse-grained representation with high fidelity. But the code itself contains less information, so ``good tracking'' means tracking a simpler target. The constrained system $B$ cannot express high-frequency structure because it never receives that information, yet it remains well-aligned with what it does receive.

This asymmetry between mismatch and complexity is precisely what the constraint hypothesis predicts: the code does not compromise \textit{alignment} (mismatch stays low) but rather restricts \textit{behavioral complexity} (effective dimensionality collapses). This is qualitatively different from prediction failure, where mismatch would increase commensurately with information loss. Critically, coupled systems maintain substantially lower mismatch than uncoupled controls ($\Delta = 0.60$) across all bandwidths.

\begin{figure}[ht]
\centering
\includegraphics[width=0.8\textwidth]{fig_mismatch.pdf}
\caption{Mismatch between systems as a function of code bandwidth. Coupled systems (solid line) maintain lower mismatch than uncoupled controls (dashed line) across all bandwidths.}
\label{fig:mismatch}
\end{figure}

\subsection{Control: Coupling Is Necessary}

To verify that the bottleneck mechanism produces these effects, we ran control simulations with $\lambda = 0$ (no coupling). Without coupling:
\begin{itemize}
\item Mismatch between $A$ and $B$ remains high ($\Delta = 0.64$) regardless of $k$
\item $B$'s complexity ($\Neff = 13.1$) is unaffected by $k$ (since $k$ is irrelevant when $\lambda = 0$)
\item Systems drift apart with no alignment
\end{itemize}
This confirms that the complexity collapse in $B$ is caused by the bandwidth-limited coupling, not by any intrinsic property of the dynamics.

\subsection{Control: Mode Structure Matters}

To test whether complexity collapse depends on Fourier structure specifically or merely on projection dimensionality, we repeated the experiment using random $k$-mode projections: instead of selecting the lowest $k$ Fourier modes, we selected $k$ modes uniformly at random from all available Fourier indices (including high-frequency modes), holding $k$ fixed.

The results differ strikingly (Figure~\ref{fig:random}). With random mode selection:
\begin{itemize}
\item $\Neff(B)$ remains high ($\approx 15$--$17$) across all $k$, showing no complexity collapse
\item Mismatch is substantially higher ($\Delta \approx 0.47$--$0.62$) than with low-frequency coupling
\item At low $k$, random projections produce \textit{higher} complexity in $B$ than in $A$
\end{itemize}

This control demonstrates that the constraining effect is not about dimensionality per se, but about the \textit{structure} of the projection. Low-frequency Fourier modes capture spatially coherent patterns---the smooth ``collective'' behavior of the oscillator lattice. Random modes mix high-frequency noise with low-frequency signal, preventing $B$ from locking onto $A$'s coherent structure. The constraint is effective precisely because low-frequency codes discard incoherent microscale detail while preserving dynamically relevant macroscale organization. Bandwidth limitation is thus necessary but not sufficient: the code must also select \textit{coherent} degrees of freedom.

\begin{figure}[ht]
\centering
\includegraphics[width=0.8\textwidth]{fig_random_control.pdf}
\caption{Comparison of low-frequency Fourier coupling (blue) versus random mode selection (red). Random projections of the same dimensionality $k$ fail to induce complexity collapse, demonstrating that the constraining effect depends on capturing spatially coherent structure, not merely on projection dimensionality. Dashed line shows $\Neff(A)$ baseline.}
\label{fig:random}
\end{figure}

Figure~\ref{fig:recon} illustrates why this difference arises. Low-frequency modes reconstruct the smooth, collective structure of the original signal; random modes produce incoherent fragments that fail to capture macroscopic organization. The responding system $B$ can lock onto a coherent target but not onto noise.

\begin{figure}[ht]
\centering
\includegraphics[width=\textwidth]{fig_reconstruction_comparison.pdf}
\caption{Visual comparison of reconstruction quality. (A) Original signal with low-frequency structure plus noise. (B) Low-frequency Fourier reconstruction ($k=4$ modes) captures the smooth collective pattern. (C) Random mode selection ($k=4$ modes) produces incoherent fragments. System $B$ can track the coherent reconstruction in (B) but not the noise in (C).}
\label{fig:recon}
\end{figure}

\subsection{Coupling Strength Effects}

Varying $\lambda$ at fixed $k = 8$ reveals a trade-off (Figure~\ref{fig:lambda}):
\begin{itemize}
\item $\lambda = 0$: No alignment; systems evolve independently ($\Delta = 0.64$)
\item $\lambda = 0.25$--$1.0$: $B$ tracks $A$ through the code; mismatch decreases ($\Delta \approx 0.36$--$0.43$)
\item $\lambda \geq 2$: Stronger coupling further reduces mismatch but increases $B$'s complexity as it more closely tracks $A$'s rich dynamics
\end{itemize}
Intermediate coupling produces the most biologically relevant regime: stable tracking with preserved internal dynamics.

\begin{figure}[ht]
\centering
\includegraphics[width=0.8\textwidth]{fig_lambda.pdf}
\caption{Effect of coupling strength $\lambda$ on system $B$'s complexity (blue, left axis) and mismatch (red, right axis) at fixed bandwidth $k = 8$. Stronger coupling reduces mismatch while allowing $B$ to express richer dynamics.}
\label{fig:lambda}
\end{figure}

\section{Discussion}

\subsection{Codes as Stabilizing Constraints}

Our results demonstrate that low-dimensional codes act as \textit{dimensional constraints} that restrict the accessible state space of coupled systems. The responding system's effective dimensionality collapses systematically with code bandwidth, while its mismatch with the driving system remains bounded. This is distinct from prediction: the code does not enable forecasting of future states, nor does it permit reconstruction of the full microstate. Rather, it shapes the dynamics of the responding system by limiting what trajectories remain accessible.

A note on terminology: we use ``code'' in the dynamical-control sense---an interface variable that defines the syntax of allowable interactions between systems---rather than in the semiotic sense of an arbitrary symbol-to-referent mapping. The Fourier projection is not merely a filter that attenuates signal; it establishes what information \textit{can} cross the boundary and thereby shapes downstream dynamics. Arbitrariness (as in genetic codes) is a special case of coding, not a requirement. What matters functionally is that the interface constrains without reconstructing.

We emphasize what the simulation demonstrates directly: \textit{complexity collapse} (reduced effective dimensionality in the constrained system) and \textit{bounded mismatch} (stable tracking of the code despite lost information). We hypothesize, but do not directly test, that such dimensionality reduction is a prerequisite for non-ergodic basin formation---the responding system explores a lower-dimensional manifold, making persistent organization possible in finite time. Whether this produces resilient attractors would require additional perturbation studies.

This distinction suggests a refinement of how biological coding structures might function. Genetic regulatory motifs, developmental programs, and homeostatic setpoints may act primarily as dimensional constraints that \textit{enable} non-ergodic organization by reducing effective state space complexity, rather than as predictive representations that \textit{compute} environmental states.

\subsection{Relation to Existing Theory}

Our framework connects to several existing theoretical traditions:

\textbf{Canalization and robustness.} Waddington's canalization \citep{waddington1942,waddington1957} and subsequent mathematical treatments \citep{kerszberg1989,wagner1996,wagner2012,siegal2002} describe how developmental trajectories become robust to perturbation. Pervasive robustness appears to be a general feature of biological systems \citep{felix2015}. Our model provides a mechanism: low-dimensional codes constrain accessible trajectories, producing canalized dynamics as a consequence of bandwidth limitation.

\textbf{Markov blankets and boundaries.} The Markov blanket formalism \citep{palacios2020,friston2013,kirchhoff2018} treats boundaries as partitions that separate internal from external states. Blankets define \textit{which} variables mediate coupling; we quantify \textit{how many degrees of freedom} can cross. The code $C_k$ is a bandwidth-limited blanket, and $k$ provides a tunable parameter that blanket formalisms do not typically specify.

\textbf{Information bottleneck.} The information bottleneck principle \citep{tishby1999,tishby2015} describes compression that preserves predictive relevance. Thermodynamic perspectives connect prediction to dissipation \citep{still2012}. Our framework differs operationally: the outcome variable is \textit{behavioral complexity and stability of $B$}, not predictive information about $A$'s future. We do not optimize $I(C_k; A_{\text{future}})$; we measure how $k$ constrains $\Neff(B)$ while bounding mismatch.

\textbf{Coarse-graining and downward causation.} Codes as dimensional projections relate to coarse-graining \citep{balduzzi2008,flack2017,shalizi2001,israeli2004}. Our contribution is showing that bandwidth-limited coarse-graining systematically constrains downstream dynamics, providing a mechanism for ``downward causation'' without invoking strong emergence.

\textbf{Computational mechanics.} The $\varepsilon$-machine formalism \citep{crutchfield1989,shalizi2001b} identifies minimal sufficient statistics for prediction. We characterize a sibling concept: \textit{minimal stabilizing interfaces}---projections that constrain downstream dynamics while potentially having low predictive power by design. Codes may be effective precisely because they discard predictively relevant but destabilizing microstate detail.

\textbf{Constraint closure.} Recent work frames biological organization as closure of constraints \citep{mossio2009,montevil2015,moreno2015,kauffman1993}. Our model provides a quantitative instantiation: coupling strength $\lambda$ determines whether the code has causal efficacy over $B$, while bandwidth $k$ sets the dimensionality of that constraint. The responding system is ``closed'' under the code in the sense that its accessible trajectories are shaped by $C_k$ rather than by $A$'s full microstate.

\textbf{Biological applications.} The mechanism we describe may illuminate coding structures across scales: gene regulatory network motifs \citep{alon2007,davidson2010} and developmental constraints \citep{arthur2004}. In biological terms, the ``coherent macroscopic degrees of freedom'' that enable constraint correspond to collective variables such as tissue-scale morphogenetic modes, homeostatic setpoints, or low-order regulatory motifs---projections that capture dynamically relevant organization while discarding molecular-scale fluctuations.

\textbf{Membranes as dimensional bottlenecks.} Consider a cell membrane: it is a physical dimensional bottleneck for coupling. The external and internal systems are coupled only through a restricted set of interface variables---receptor occupancy, ion-channel conductance, transmembrane voltage---so the effective dimensionality of cross-boundary influence is far smaller than the dimensionality of either system's microstate. System $B$ ``listens'' because partial coupling to the environment is necessary for viability (resource detection, threat response), yet attempting to track high-dimensional microstate detail would be metabolically prohibitive. In the language of this model, the membrane implements a bandwidth-limited code $C_k$ where evolution has tuned both $k$ (interface dimensionality) and structure (which degrees of freedom couple) to maximize actionable environmental information per unit metabolic cost. Crucially, as our random-mode control illustrates, dimensionality restriction alone is insufficient; biological interfaces are structured to transmit coherent, actionable degrees of freedom rather than arbitrary mixtures of microstate detail.

\subsection{What This Paper Is Not Claiming}

We do not claim that codes enable prediction of high-dimensional biological dynamics. We also do not claim that the complexity collapse we observe directly produces stable attractors or non-ergodic basins; rather, we hypothesize that dimensionality reduction is a \textit{prerequisite} for such organization, which would require perturbation studies to verify. Finally, we do not claim that this mechanism explains all biological coding; rather, we provide a minimal model demonstrating that bandwidth-limited coupling produces systematic complexity collapse while preserving bounded tracking.

\section{Conclusion}

We have shown that low-dimensional codes can function as dimensional constraints between coupled high-dimensional systems. When a responding system couples to a driving system through a bandwidth-limited code, the responding system exhibits systematic complexity collapse: its effective dimensionality scales with code bandwidth while maintaining bounded tracking error. Control experiments with random mode projections demonstrate that the constraining effect depends on capturing spatially coherent structure, not merely on projection dimensionality.

These results suggest a mechanism by which biological coding structures---genetic regulatory motifs, developmental programs, homeostatic setpoints---might restrict accessible state space and thereby enable persistent organization. The key contribution is demonstrating that low-dimensional coupling induces complexity collapse without commensurate tracking failure, providing a quantitative instantiation of how constraints can shape dynamics without serving as predictive representations. Future work should test whether such dimensional constraints produce resilient attractor basins under perturbation.

\section*{Acknowledgments}

[To be added]

\section*{Declaration of generative AI use in the writing process}

During the preparation of this work the author used Claude (Anthropic) for manuscript drafting, editing, and code development. The author reviewed and edited the content as needed and takes full responsibility for the content of the published article.

\section*{CRediT author statement}

\textbf{Ian Todd:} Conceptualization, Methodology, Software, Formal analysis, Investigation, Writing -- Original Draft, Writing -- Review \& Editing, Visualization.

\section*{Data availability}

All simulation code and data are publicly available at \url{https://github.com/todd866/code-formation-jtb}.

\bibliography{references}

\end{document}
