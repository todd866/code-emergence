\documentclass[11pt]{letter}
\usepackage[margin=1in]{geometry}
\usepackage{hyperref}

\signature{Ian Todd\\Sydney Medical School\\University of Sydney}
\address{Sydney Medical School\\University of Sydney\\Sydney, NSW 2006\\Australia\\itod2305@uni.sydney.edu.au}

\begin{document}

\begin{letter}{Editorial Office\\Information Geometry\\Springer}

\opening{Dear Editors,}

I am pleased to submit the manuscript entitled ``Quotient Geometry of Statistical Manifolds Under Dimensional Collapse'' for consideration in \textit{Information Geometry}.

This paper develops the differential geometry of statistical manifolds under collapse maps---smooth maps $\pi: \mathcal{M} \to \mathbb{R}^k$ with $k < \dim(\mathcal{M})$. The work generalizes the minimal embedding theorem for recurrent processes (manuscript INGE-D-25-00099, currently under review at this journal) to arbitrary dimensional reduction on statistical manifolds.

The main contributions are:

\begin{enumerate}
    \item \textbf{Fiber Structure Theorem:} We show that collapse maps foliate $\mathcal{M}$ into fibers along which the Fisher metric degenerates. Points on the same fiber are statistically non-identifiable, providing a differential-geometric interpretation of Watanabe's singular learning theory.

    \item \textbf{Quotient Metric Theorem:} We characterize when the Fisher metric descends to a well-defined Riemannian metric on the quotient $\mathcal{M}/\sim_\pi$, establishing that fibers must be totally geodesic. We show how $\alpha$-connections transform under collapse and identify conditions for dual flatness to survive.

    \item \textbf{Covering Number Bounds:} We prove that the number of $\varepsilon$-distinguishable equivalence classes scales as $N(\varepsilon) \sim \varepsilon^{-r}$ where $r$ is the projection rank, quantifying the ``discretization'' induced by dimensional collapse.
\end{enumerate}

The paper connects to core themes of information geometry: Chentsov's uniqueness theorem (the Fisher metric is invariant under sufficient statistics; our quotient construction respects this), Amari's $\alpha$-geometry (we characterize when dual connections descend), and Watanabe's singular learning theory (our fiber structure provides the differential-geometric setting for his algebraic singularities).

This work is appropriate for \textit{Information Geometry} because it develops fundamental Riemannian and affine geometry on statistical manifolds, extending classical results on submersions and quotients to the information-geometric setting.

The manuscript has not been submitted elsewhere. There are no conflicts of interest to declare.

Thank you for considering this submission.

\closing{Sincerely,}

\end{letter}
\end{document}
