\documentclass[11pt]{letter}
\usepackage[margin=1in]{geometry}
\usepackage{hyperref}

\signature{Ian Todd\\Sydney Medical School\\University of Sydney}
\address{Sydney Medical School\\University of Sydney\\Sydney, NSW, Australia\\itod2305@uni.sydney.edu.au}

\begin{document}

\begin{letter}{Editorial Office\\Progress in Biophysics \& Molecular Biology}

\opening{Dear Editors,}

I am pleased to submit the manuscript ``The Code-Constraint Problem in Biological Systems: How Low-Dimensional Interfaces Shape High-Dimensional Dynamics'' for consideration as a Perspective article in \textit{Progress in Biophysics \& Molecular Biology}.

\textbf{Summary.} Biological systems operate in high-dimensional state spaces, yet experimental readouts---order parameters, expression markers, population codes---are inevitably low-dimensional. This dimensional mismatch creates systematic interpretive challenges across molecular, cellular, and systems biology. The manuscript proposes a unifying framework: low-dimensional interfaces function as \textit{stabilizing constraints} rather than information channels. Using coupled oscillator simulations, we demonstrate a distinctive signature---complexity collapse in responding systems with bounded tracking---that appears only for structured projections capturing coherent collective variables.

\textbf{Why PBMB?} This work bridges multiple domains that your journal integrates:
\begin{itemize}
\item \textit{Molecular biophysics}: We connect to protein conformational dynamics and the reliability of order-parameter descriptions (folding funnels, reaction coordinates).
\item \textit{Systems biology}: The framework addresses when coarse-grained models are reliable versus when they become ``shadows'' of latent dynamics---a question central to gene regulatory network modeling.
\item \textit{Bioelectricity and morphogenesis}: We explicitly connect to Michael Levin's work on bioelectric control of pattern formation, where low-dimensional voltage gradients constrain high-dimensional cellular dynamics.
\item \textit{Theoretical synthesis}: The paper positions the constraint framework relative to synergetics, information bottleneck theory, and Markov blanket formalism, offering a new quantitative dimension (bandwidth) to these established approaches.
\end{itemize}

\textbf{Novel contribution.} The central claim is testable and distinct from existing frameworks: we propose a \textit{dynamical regime diagnostic}---complexity collapse in responding systems while tracking error remains bounded with respect to coarse-grained reconstruction. This signature discriminates effective biological codes (structured collective variables) from arbitrary dimensional reductions (which fail to induce constraint). The minimal model is intentionally abstract to emphasize generality; we include a concrete translation to protein-ligand binding with experimentally testable predictions.

\textbf{Broader significance.} The framework offers practical criteria for evaluating when low-dimensional models are reliable, with implications for experimental design across scales---from choosing reaction coordinates in molecular dynamics to interpreting single-cell manifolds to understanding neural population constraints.

I confirm that this work is original, has not been published elsewhere, and is not under consideration at another journal. All simulation code and data are publicly available at \url{https://github.com/todd866/code-formation-jtb}.

Thank you for considering this submission. I look forward to your response.

\closing{Sincerely,}

\end{letter}
\end{document}
